\documentclass{beamer}

\mode<presentation> { \usetheme{gruvbox} }
\setbeamerfont{frametitle}{size=\huge}

\usepackage{graphicx} % Allows including images
\usepackage{booktabs} % Allows the use of \toprule, \midrule and \bottomrule in tables
%\usepackage{listings}             % Include the listings-package
\usepackage{minted}
\usepackage{tikz}
\usepackage{drawstack}
\usetikzlibrary{calc,shapes.callouts,shapes.arrows,chains,positioning,fit,shapes, arrows.meta, arrows}
\usepackage{verbatimbox}
\usepackage{tcolorbox}
\usepackage{forloop}
\usepackage{seqsplit}

\usemintedstyle{paraiso-dark}
\graphicspath{ {./images/}{../../slides-source-dep/images/} }
\DeclareGraphicsExtensions{.png,.pdf}

\newcommand{\pointthis}[2]{
    \tikz[remember picture,baseline]{\node[anchor=base,inner sep=0,outer sep=0]%
    (#1) {\underline{#1}};\node[overlay,rectangle callout,%
    callout relative pointer={(0.2cm,0.7cm)},fill=green!50] at ($(#1.north)+(-.5cm,-1.4cm)$) {#2};}%
}%

\newcounter{loopcntr}
\newcommand{\rpt}[2][1]{%
  \forloop{loopcntr}{0}{\value{loopcntr}<#1}{#2}%
}

\newcommand{\hash}[1]{{\ttfamily\seqsplit{#1}}}

\newenvironment{zerohyphen}
 {\global\chardef\savedhyphenchar=\hyphenchar\font % save the current hyphenchar
  \lefthyphenmin=1 \righthyphenmin=1 % no limits on hyphenation
  \hyphenchar\font=23 }
 {\par\hyphenchar\font=\savedhyphenchar}% eject the paragraph and restore

%----------------------------------------------------------------------------------------
%	TITLE PAGE
%----------------------------------------------------------------------------------------

\title[Introduction to Web Exploitation]{\huge \textbf{Introduction to Web Exploitation}} % The short title appears at the bottom of every slide, the full title is only on the title page

\author{Andrew Haberlandt} % Your name
\date{January 2021} % Date, can be changed to a custom date

\begin{document}

{ % this brace groups the background template with just the first slide
\usebackgroundtemplate{%
    \begin{tikzpicture}
        \path [outer color = blue!5, inner color = blue!1]
        (0,0) rectangle (\paperwidth,\paperheight);
        \node[anchor=south west, inner sep=0,line width=0,draw,text width=\paperwidth,fill=almostblack] at (0,0) {\textcolor{darkgray}{\hash{00110110100011010001101101101111100010010111011001110101001111101110100101100101001000001010111000001100110000111010100100001110100010110101001010100001011100000110011111111111100110011010100100101111110111110011101011110010100001001101101010111111011000010110011101110110110000000101101101111110111010111001110010111100110110100100111110111011010110111010101101000011000100011101001011010101111101010000001000001111011111000100111001010000101010000010111001101111101111011111011001110100101101100000011000011110110111010111110001111001110011011110101001001011110001101010110000110110100011000101110110101001110100011101100111101000001001111110000111100010010110000111110101010100000000001110000001010001110110001111000100001100010110101011100011101110101100111111010111101000100111000011110110100011000110111011001101101111111100001010101010010100100101110101011111010110100111011000101010111000010110010010011000010110011000111000110110000010110001100110100011000000111100110110101011100100011110100011011010001101000110110110111110001001011101100111010100111110111010010110010100100000101011100000110011000011101010010000111010001011010100101010000101110000011001111111111110011001101010010010111111011111001110101111001010000100110110101011111101100001011001110111011011000000010110110111111011101011100111001011110011011010010011111011101101011011101010110100001100010001110100101101010111110101000000100000111101111100010011100101000010101000001011100110111110111101111101100111010010110110000001100001111011011101011111000111100111001101111010100100101111000110101011000011011010001100010111011010100111010001110110011110100000100111111000011110001001011000011111010101010000000000111000000101000111011000111100010000110001011010101110001110111010110011111101011110100010011100001111011010001100011011101100110110111111110000101010101001010010010111010101111101011010011101100010101011100001011001001001100001011001100011100011011000001011000110011010001100000011110011011010101110010001111010}}};
    \end{tikzpicture}
}


\begin{frame}
    \titlepage % Print the title page as the first slide
    \begin{tikzpicture}
        \node[anchor=south west] at (20, 20) {\includegraphics[width=0.15\paperwidth]{logo.png}};
    \end{tikzpicture}
\end{frame}

\begin{frame}
    \frametitle{Opportunities this week}
    \begin{itemize}
        \item{Women in Cybersecurity Meeting, Thursday @ 7pm}
        \begin{itemize}
            \item{$<$insert topic for this week$>$}
            \item{$<$insert website/meeting link / mailing list$>$}
        \end{itemize}
        \item{Cyber Security Club Bootcamp CTF continues...}
    \end{itemize}
\end{frame}

} % end background template
\setbeamercolor{background canvas}{bg=almostblack}
\setbeamertemplate{section in toc}[square]
\begin{frame}
    \frametitle{Overview} % Table of contents slide, comment this block out to remove it
    \tableofcontents % Throughout your presentation, if you choose to use \section{} and \subsection{} commands, these will automatically be printed on this slide as an overview of your presentation
\end{frame}

\section{What \it{is} the 'Web' Category?}

\subsection{A little about the Internet}
\begin{frame}
    \frametitle{HTTP}
    \begin{itemize}
        \item You want to go to http://google.com \ldots How?
    \end{itemize}
    \begin{enumerate}
        \item DNS lookup for google.com $\rightarrow$ IP address
        \item Open a TCP connection, send an HTTP request message
        \item Parse HTTP response message
        \item Parse and render the HTML in the response
        \begin{itemize}
            \item Sometimes this requires additional requests for external%
            resources, which goes back to \#1
        \end{itemize}
    \end{enumerate}
%    \begin{tikzpicture}
%	\node (computer) [draw, double] at (0, 0) {Your Computer};
%	\node (server) [draw, outer sep=40pt, right of=computer,circle] at (1, 0) {Server};
%        \draw [->] (computer) edge (server) (server) edge (computer);
%    \end{tikzpicture}
\end{frame}


\subsection{What does a 'vulnerability' look like in a Web application?}
\begin{frame}
    \frametitle{What \it{is} the 'Web' Category?}
    Types of vulnerabilities to consider
    \begin{itemize}
        \item \textbf{Sensitive data exposure / information leakage:} Can you get%
            the server to give you information you shouldn't have access to?
        \item \textbf{Broken Access Control:} Can you modify data on the server
            without proper authorization?
        \item \textbf{Broken Authentication:} Can you login as another user or
            compromise their password or session tokens?
        \item \textbf{Manipulating Responses to other users:} Can you modify
            resources provided to other users in a way that will directly
            or indirectly give you access to their account?
    \end{itemize}
\end{frame}

\subsection{Why do we care?}
\begin{frame}
    \frametitle{Why do we care?}
    \begin{itemize}
        \item There are lots of websites, with lots of data
        \item APIs
        \item Web technologies bleeding into desktop apps (Electron)
        \item Bug bounties are largely web applications (e.g. HackerOne, Bugcrowd)
    \end{itemize}
\end{frame}

\section{Demo: Using Chrome 'Developer Tools' to Hack Stuff}
\begin{frame}
    \frametitle{Demo: Chrome Developer Tools} 
\end{frame}

\subsection{HTML \& Inspect Element}
\subsection{Javascript \& the Console}
\subsection{Note about Frameworks}
\subsection{The Debugger}
\subsection{The 'Network' Tab}
\subsection{Storage}

\section{Useful Tools and What They Do}
\begin{frame}
    \frametitle{Tools}
\end{frame}

\end{document}
