\documentclass{beamer}

\mode<presentation> { \usetheme{gruvbox} }
\setbeamerfont{frametitle}{size=\huge}

\usepackage{graphicx} % Allows including images
\usepackage{booktabs} % Allows the use of \toprule, \midrule and \bottomrule in tables
%\usepackage{listings}             % Include the listings-package
\usepackage{minted}
\usepackage{tikz}
\usepackage{drawstack}
\usetikzlibrary{calc,shapes.callouts,shapes.arrows,chains,positioning,fit,shapes, arrows.meta, arrows}
\usepackage{verbatimbox}
\usepackage{tcolorbox}
\usepackage{forloop}
\usepackage{seqsplit}

\usemintedstyle{paraiso-dark}
\graphicspath{ {./images/} }
\DeclareGraphicsExtensions{.png,.pdf}

\newcommand{\pointthis}[2]{
    \tikz[remember picture,baseline]{\node[anchor=base,inner sep=0,outer sep=0]%
    (#1) {\underline{#1}};\node[overlay,rectangle callout,%
    callout relative pointer={(0.2cm,0.7cm)},fill=green!50] at ($(#1.north)+(-.5cm,-1.4cm)$) {#2};}%
}%

\newcounter{loopcntr}
\newcommand{\rpt}[2][1]{%
  \forloop{loopcntr}{0}{\value{loopcntr}<#1}{#2}%
}

\newcommand{\hash}[1]{{\ttfamily\seqsplit{#1}}}

\newenvironment{zerohyphen}
 {\global\chardef\savedhyphenchar=\hyphenchar\font % save the current hyphenchar
  \lefthyphenmin=1 \righthyphenmin=1 % no limits on hyphenation
  \hyphenchar\font=23 }
 {\par\hyphenchar\font=\savedhyphenchar}% eject the paragraph and restore

%----------------------------------------------------------------------------------------
%	TITLE PAGE
%----------------------------------------------------------------------------------------

\title[Example RE Slide Deck]{\huge \textbf{Example RE Slide Deck:} \\ Some Subtitle - Code in ~C} % The short title appears at the bottom of every slide, the full title is only on the title page

\author{Andrew Haberlandt} % Your name
\date{\today} % Date, can be changed to a custom date

\begin{document}

{ % this brace groups the background template with just the first slide
\usebackgroundtemplate{%
    \begin{tikzpicture}
        \path [outer color = blue!5, inner color = blue!1]
        (0,0) rectangle (\paperwidth,\paperheight);
        \node[anchor=south west, inner sep=0,line width=0,draw,text width=\paperwidth,fill=almostblack] at (0,0) {\textcolor{darkgray}{\hash{00110110100011010001101101101111100010010111011001110101001111101110100101100101001000001010111000001100110000111010100100001110100010110101001010100001011100000110011111111111100110011010100100101111110111110011101011110010100001001101101010111111011000010110011101110110110000000101101101111110111010111001110010111100110110100100111110111011010110111010101101000011000100011101001011010101111101010000001000001111011111000100111001010000101010000010111001101111101111011111011001110100101101100000011000011110110111010111110001111001110011011110101001001011110001101010110000110110100011000101110110101001110100011101100111101000001001111110000111100010010110000111110101010100000000001110000001010001110110001111000100001100010110101011100011101110101100111111010111101000100111000011110110100011000110111011001101101111111100001010101010010100100101110101011111010110100111011000101010111000010110010010011000010110011000111000110110000010110001100110100011000000111100110110101011100100011110100011011010001101000110110110111110001001011101100111010100111110111010010110010100100000101011100000110011000011101010010000111010001011010100101010000101110000011001111111111110011001101010010010111111011111001110101111001010000100110110101011111101100001011001110111011011000000010110110111111011101011100111001011110011011010010011111011101101011011101010110100001100010001110100101101010111110101000000100000111101111100010011100101000010101000001011100110111110111101111101100111010010110110000001100001111011011101011111000111100111001101111010100100101111000110101011000011011010001100010111011010100111010001110110011110100000100111111000011110001001011000011111010101010000000000111000000101000111011000111100010000110001011010101110001110111010110011111101011110100010011100001111011010001100011011101100110110111111110000101010101001010010010111010101111101011010011101100010101011100001011001001001100001011001100011100011011000001011000110011010001100000011110011011010101110010001111010}}};
    \end{tikzpicture}
}


\begin{frame}
    \titlepage % Print the title page as the first slide
    \begin{tikzpicture}
        \node[anchor=south west] at (20, 20) {\includegraphics[width=0.15\paperwidth]{logo.png}};
    \end{tikzpicture}
\end{frame}

\begin{frame}
    \frametitle{Opportunities this week}
    \begin{itemize}
        \item{Women in Cybersecurity Meeting, Thursday @ 7pm}
        \begin{itemize}
            \item{$<$insert topic for this week$>$}
            \item{$<$insert website/meeting link / mailing list$>$}
        \end{itemize}
        \item{Cyber Security Club Bootcamp CTF continues...}
    \end{itemize}
\end{frame}

} % end background template
\setbeamercolor{background canvas}{bg=almostblack}
\setbeamertemplate{section in toc}[square]
\begin{frame}
    \frametitle{Overview} % Table of contents slide, comment this block out to remove it
    \tableofcontents % Throughout your presentation, if you choose to use \section{} and \subsection{} commands, these will automatically be printed on this slide as an overview of your presentation
\end{frame}

\section{An Example Section of the Presentation}

\begin{frame}
    \frametitle{An example first slide}
    Some RE stuff:
    \begin{itemize}
        \item A thing
        \item Another thing
    \end{itemize}
\end{frame}

\section{Another Example Section of the Presentation}

\begin{frame}
    \frametitle{An example of some stack frame stuff}
    \begin{tikzpicture}[scale=.5,draw=lightred,text=invtext]
        \tiny % can also use any from https://tex.stackexchange.com/questions/107057/adjusting-font-size-with-tikz-picture
        \stacktop{}
        \startframe
        \cell{char input[32]} \cellcom{RBP - 64}
        \cell{int * \textit{c}} \cellcom{RBP - 32}
        \cell{int \textit{b}} \cellcom{RBP - 24}
        \cell{int \textit{a}} \cellcom{RBP - 16}
        \cell{Stack Canary} \cellcom{RBP - 8}
        \cell{Saved RBP} \cellptr{RBP, RSP}
        \bcell{Saved RIP} \cellcom{RBP + 8}
        \finishframe{function \\ {\tt foo ()}}
        \startframe
        \cell{Stack Canary} \cellcom{RBP - 8}
        \cell{Saved RBP} \cellptr{RBP, RSP}
        \bcell{Saved RIP} \cellcom{RBP + 8}
        \finishframe{function \\ {\tt main ()}}
        \stackbottom{}
    \end{tikzpicture}
\end{frame}

\begin{frame}[fragile]
    \frametitle{How Could That Be Exploited?}
    \begin{columns}[c]

        \column{.5\textwidth}
        \textbf{Example Code} \\
        \begin{minted}[gobble=12]{c}
            int main(void) {
                int authorized[2];
                int username[32];
                authorized[0] = 0x0
                for (int i = 0;
                i <= 32;
                i++) {
                    username[i] = 0x1
                }
            }
        \end{minted}

        \column{.5\textwidth}
        \hspace{-1.8cm} % negative horizontal space to align properly
        \begin{tikzpicture}[scale=.5, draw=lightred, text=invtext]
            \tiny
            \stacktop{}
            \startframe
            \cell{Stack} \coordinate (stack) at (currentcell.east);
            \cell{\ldots}
            \cell{Heap}
            \cell{Data}
            \cell{Text}
            \finishframe{\textbf{Program's \\ memory space}}
            \drawstruct{(5,3)}
            \structcell{int authorized[2]}
            \structcell{int username[32]}
            \structcell{\ldots} \coordinate(O1) at (currentcell.west);

            \draw[->] (O1) .. controls (O1 |- stack) .. (stack);
        \end{tikzpicture}
    \end{columns}
\end{frame}

\end{document}
