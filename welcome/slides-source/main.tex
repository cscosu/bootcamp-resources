\documentclass{beamer}

\mode<presentation> { \usetheme{gruvbox} }
\setbeamerfont{frametitle}{size=\huge}

\usepackage{graphicx} % Allows including images
\usepackage{booktabs} % Allows the use of \toprule, \midrule and \bottomrule in tables
%\usepackage{listings}             % Include the listings-package
\usepackage{minted}
\usepackage{tikz}
\usepackage{drawstack}
\usetikzlibrary{calc,shapes.callouts,shapes.arrows,chains,positioning,fit,shapes, arrows.meta, arrows}
\usepackage{verbatimbox}
\usepackage{tcolorbox}
\usepackage{forloop}
\usepackage{seqsplit}

\usemintedstyle{paraiso-dark}
\graphicspath{ {./images/}{../../slides-source-dep/images/} }
\DeclareGraphicsExtensions{.png,.pdf}

\newcommand{\pointthis}[2]{
    \tikz[remember picture,baseline]{\node[anchor=base,inner sep=0,outer sep=0]%
    (#1) {\underline{#1}};\node[overlay,rectangle callout,%
    callout relative pointer={(0.2cm,0.7cm)},fill=green!50] at ($(#1.north)+(-.5cm,-1.4cm)$) {#2};}%
}%

\newcounter{loopcntr}
\newcommand{\rpt}[2][1]{%
  \forloop{loopcntr}{0}{\value{loopcntr}<#1}{#2}%
}

\newcommand{\hash}[1]{{\ttfamily\seqsplit{#1}}}

\newenvironment{zerohyphen}
 {\global\chardef\savedhyphenchar=\hyphenchar\font % save the current hyphenchar
  \lefthyphenmin=1 \righthyphenmin=1 % no limits on hyphenation
  \hyphenchar\font=23 }
 {\par\hyphenchar\font=\savedhyphenchar}% eject the paragraph and restore

%----------------------------------------------------------------------------------------
%	TITLE PAGE
%----------------------------------------------------------------------------------------

\title[Welcome Back - Spring 2021]{\huge \textbf{Welcome Back - Spring 2021}} % The short title appears at the bottom of every slide, the full title is only on the title page

\author{Kyle Westhaus} % Your name
\date{January 12 2021} % Date, can be changed to a custom date

\begin{document}

{ % this brace groups the background template with just the first slide
\usebackgroundtemplate{%
    \begin{tikzpicture}
        \path [outer color = blue!5, inner color = blue!1]
        (0,0) rectangle (\paperwidth,\paperheight);
        \node[anchor=south west, inner sep=0,line width=0,draw,text width=\paperwidth,fill=almostblack] at (0,0) {\textcolor{darkgray}{\hash{00110110100011010001101101101111100010010111011001110101001111101110100101100101001000001010111000001100110000111010100100001110100010110101001010100001011100000110011111111111100110011010100100101111110111110011101011110010100001001101101010111111011000010110011101110110110000000101101101111110111010111001110010111100110110100100111110111011010110111010101101000011000100011101001011010101111101010000001000001111011111000100111001010000101010000010111001101111101111011111011001110100101101100000011000011110110111010111110001111001110011011110101001001011110001101010110000110110100011000101110110101001110100011101100111101000001001111110000111100010010110000111110101010100000000001110000001010001110110001111000100001100010110101011100011101110101100111111010111101000100111000011110110100011000110111011001101101111111100001010101010010100100101110101011111010110100111011000101010111000010110010010011000010110011000111000110110000010110001100110100011000000111100110110101011100100011110100011011010001101000110110110111110001001011101100111010100111110111010010110010100100000101011100000110011000011101010010000111010001011010100101010000101110000011001111111111110011001101010010010111111011111001110101111001010000100110110101011111101100001011001110111011011000000010110110111111011101011100111001011110011011010010011111011101101011011101010110100001100010001110100101101010111110101000000100000111101111100010011100101000010101000001011100110111110111101111101100111010010110110000001100001111011011101011111000111100111001101111010100100101111000110101011000011011010001100010111011010100111010001110110011110100000100111111000011110001001011000011111010101010000000000111000000101000111011000111100010000110001011010101110001110111010110011111101011110100010011100001111011010001100011011101100110110111111110000101010101001010010010111010101111101011010011101100010101011100001011001001001100001011001100011100011011000001011000110011010001100000011110011011010101110010001111010}}};
    \end{tikzpicture}
}


\begin{frame}
    \titlepage % Print the title page as the first slide
    \begin{tikzpicture}
        \node[anchor=south west] at (20, 20) {\includegraphics[width=0.15\paperwidth]{logo.png}};
    \end{tikzpicture}
\end{frame}
} % end background template

%\begin{frame}
%    \frametitle{Opportunities this week}
%    \begin{itemize}
%        \item{Women in Cybersecurity Meeting, Thursday @ 7pm}
%        \begin{itemize}
%            \item{$<$insert topic for this week$>$}
%            \item{$<$insert website/meeting link / mailing list$>$}
%        \end{itemize}
%        \item{Cyber Security Club Bootcamp CTF continues...}
%    \end{itemize}
%\end{frame}

\setbeamercolor{background canvas}{bg=almostblack}
\setbeamertemplate{section in toc}[square]
%\begin{frame}
%    \frametitle{Overview} % Table of contents slide, comment this block out to remove it
%    \tableofcontents % Throughout your presentation, if you choose to use \section{} and \subsection{} commands, these will automatically be printed on this slide as an overview of your presentation
%\end{frame}

\section{Reminders}

\subsection{Officers}
\begin{frame}
    \frametitle{Reminders - Officers}
    \begin{itemize}
        \item Kyle Westhaus, President
        \item Yu-Shiang Jeng, Vice President
        \item Andrew Haberlandt, Treasurer
        \item Jackson Leverett, CTF Captain
        \item Nick Kuhl, Outreach Coordinator
        \item Andrew Migot, Research Coordinator
    \end{itemize}
\end{frame}


\subsection{Communication/Links}
\begin{frame}
    \frametitle{Reminders - Communication/Links}
    \begin{itemize}
        \item \textbf{Attendance} (because the College of Engineering and OSU make us): http://attend.osucyber.club
        \item \textbf{Mailing List:} http://mailinglist.osucyber.club
        \item \textbf{Slack:} http://slack.osucyber.club
        \item \textbf{Website:} http://osucyber.club
        \item \textbf{Wiki:} http://wiki.osucyber.club
    \end{itemize}
\end{frame}

\section{Bootcamp Explanation}

\subsection{Feedback Results}
\begin{frame}
    \frametitle{Feedback Results}
    \begin{itemize}
        \item CTF-specific survey, club survey
        \item People wanted:
        \begin{itemize}
            \item Better cohesion between meetings
            \item More hands-on experiences
            \item Preparation for the competition, not just the competition itself 
        \end{itemize}
    \end{itemize}
\end{frame}

\subsection{Our Past CTFs}
\begin{frame}
    \frametitle{Our Past CTFs}
    \begin{itemize}
        \item Weekend-long
        \item Range of challenge difficulty levels
        \item \dots but little background info about how to prepare an environment and begin competing
    \end{itemize}
\end{frame}

\subsection{The Plan}
\begin{frame}
    \frametitle{The Plan}
    \begin{itemize}
        \item Floated idea of "Bootcamp" in winter break feedback - something other clubs have done before (Purdue)
        \item Series of meetings dedicated to teaching those with minimal CTF experience how to be successful
        \begin{itemize}
            \item CTF background info
            \item Technical knowledge for challenge categories
            \item Wiki content about environment/tool setup
            \item Challenges hosted all semester!
        \end{itemize}
    \end{itemize}
\end{frame}

\subsection{This Semester}
\begin{frame}
    \frametitle{This Semester} 
    \begin{itemize}
        \item Tonight: background on CTFs
        \item Next week: more Bootcamp CTF specifics, start of actual bootcamp content
        \item Rest of semester: more technical education content for challenge categories, you participate and use what you learn in meeting to complete more and more challenges and get prizes along the way!
    \end{itemize}
\end{frame}

\section{CTF Background}

\subsection{Capture the Flag}
\begin{frame}
    \frametitle{Capture the Flag}
    \begin{itemize}
        \item "flags" = short snippets of identifiable text. Examples:
        \begin{itemize}
            \item $ \texttt{osuctf\{theflagincludesthestuffbeforethebraces\}} $
            \item $ \texttt{osuctf\{th15\_15\_n0t\_a\_r34l\_flag\}} $
            \item $ \texttt{osuctf\{l33tsp34k\}} $
        \end{itemize}
        \item Not so much "find this text hidden somewhere on a computer", but rather "complete this technical challenge, the flag will be apparent at the end". Example:
        \begin{itemize}
            \item "This ciphertext is weakly encoded, decode it to get the plaintext flag"
        \end{itemize}
    \end{itemize}
\end{frame}

\subsection{Why CTF?}
\begin{frame}
    \frametitle{Why CTF?}
    Very popular for college clubs like our own because:
    \begin{itemize}
        \item Very effective way to learn
        \item Very effective way to demonstrate skills to future employers
        \item Fun competition
    \end{itemize}
\end{frame}

\subsection{CTF Format}
\begin{frame}
    \frametitle{CTF Format}
    \begin{itemize}
        \item Mostly jeopardy
        \begin{itemize}
            \item Many problems, grouped by category, each has an associated number of points, available for a certain amount of time
            \item Solve a challenge, submit flag, receive points
            \item Team with most points at end of competition wins
            \item Sometimes additional rules (e.g. only a few challenges unlocked at start, first team to solve gets to pick next unlocked challenge)
        \end{itemize}
        \item King of the hill
        \item Red/blue team
    \end{itemize}

\end{frame}

\subsection{CTF Interface}
\begin{frame}
    \frametitle{CTF Interface}
    \begin{itemize}
        \item Usually a website for registration, team creation, list of challenges, flag submission, scoreboard
        \item Extremely common software to run this part of a CTF is CTFd
    \end{itemize}
\end{frame}

\subsection{Interfacing with Challenges}
\begin{frame}
    \frametitle{Interfacing with Challenges}
    \begin{itemize}
        \item Downloadable file in challenge description
        \begin{itemize}
            \item Demo base64, open in text editor
            \item Demo Somewhat Questionable LLC, open in browser
            \item Demo speedrun3, view source, connect with netcat
        \end{itemize}
    \end{itemize}
\end{frame}

\subsection{Need for CTF Tools}
\begin{frame}
    \frametitle{Need for CTF Tools}
    When using any standard operating system, you can automatically do the following with challenges
    \begin{itemize}
        \item Downloadable
        \begin{itemize}
            \item Look at file in file explorer, maybe extract, open in text editor
        \end{itemize}
        \item Service URL
        \begin{itemize}
            \item Open in web browser, something like netcat
        \end{itemize}
    \end{itemize}
    \dots but often times you need to do more! Solution:
    \begin{itemize}
        \item Use recommended tools (see the wiki!)
        \item Do your own searching to find helpful tools
        \item Write your own code
    \end{itemize}
\end{frame}

\subsection{Tool Examples}
\begin{frame}
    \frametitle{Tool Examples}
    \begin{itemize}
        \item Web tool (cryptii)
        \item Downloadable program with GUI (ghidra)
        \item Downloadable terminal program (urlbuster)
        \item Python library (pwntools)
    \end{itemize}
\end{frame}

\subsection{How do I find CTFs?}
\begin{frame}
    \frametitle{How do I find CTFs?}
    \begin{itemize}
        \item We are hosting a semester-long one!
        \item Other universities run them!
        \item Conferences run them!
        \item CTF teams host their own!
    \end{itemize}
    \dots and they are all on a list at https://ctftime.org/ (DEMO)
\end{frame}

\subsection{Other Content}
\begin{frame}
    \frametitle{Other Content}
    \begin{itemize}
        \item Livestreams/youtube videos of people solving CTFs
        \item CTF challenge solution writeups (github or blogs)
    \end{itemize}
\end{frame}

\section{Wrap Up}

\begin{frame}
    \frametitle{Wrap Up}
    \begin{itemize}
        \item Questions for now
        \item Slack for questions later
        \item Come back next week!
    \end{itemize}
\end{frame}

\end{document}
